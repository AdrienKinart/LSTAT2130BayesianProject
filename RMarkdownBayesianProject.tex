% Options for packages loaded elsewhere
\PassOptionsToPackage{unicode}{hyperref}
\PassOptionsToPackage{hyphens}{url}
%
\documentclass[
]{article}
\usepackage{lmodern}
\usepackage{amssymb,amsmath}
\usepackage{ifxetex,ifluatex}
\ifnum 0\ifxetex 1\fi\ifluatex 1\fi=0 % if pdftex
  \usepackage[T1]{fontenc}
  \usepackage[utf8]{inputenc}
  \usepackage{textcomp} % provide euro and other symbols
\else % if luatex or xetex
  \usepackage{unicode-math}
  \defaultfontfeatures{Scale=MatchLowercase}
  \defaultfontfeatures[\rmfamily]{Ligatures=TeX,Scale=1}
\fi
% Use upquote if available, for straight quotes in verbatim environments
\IfFileExists{upquote.sty}{\usepackage{upquote}}{}
\IfFileExists{microtype.sty}{% use microtype if available
  \usepackage[]{microtype}
  \UseMicrotypeSet[protrusion]{basicmath} % disable protrusion for tt fonts
}{}
\makeatletter
\@ifundefined{KOMAClassName}{% if non-KOMA class
  \IfFileExists{parskip.sty}{%
    \usepackage{parskip}
  }{% else
    \setlength{\parindent}{0pt}
    \setlength{\parskip}{6pt plus 2pt minus 1pt}}
}{% if KOMA class
  \KOMAoptions{parskip=half}}
\makeatother
\usepackage{xcolor}
\IfFileExists{xurl.sty}{\usepackage{xurl}}{} % add URL line breaks if available
\IfFileExists{bookmark.sty}{\usepackage{bookmark}}{\usepackage{hyperref}}
\hypersetup{
  pdftitle={LSTAT2130BayesianProject},
  pdfauthor={Adrien Kinart},
  hidelinks,
  pdfcreator={LaTeX via pandoc}}
\urlstyle{same} % disable monospaced font for URLs
\usepackage[margin=1in]{geometry}
\usepackage{color}
\usepackage{fancyvrb}
\newcommand{\VerbBar}{|}
\newcommand{\VERB}{\Verb[commandchars=\\\{\}]}
\DefineVerbatimEnvironment{Highlighting}{Verbatim}{commandchars=\\\{\}}
% Add ',fontsize=\small' for more characters per line
\usepackage{framed}
\definecolor{shadecolor}{RGB}{248,248,248}
\newenvironment{Shaded}{\begin{snugshade}}{\end{snugshade}}
\newcommand{\AlertTok}[1]{\textcolor[rgb]{0.94,0.16,0.16}{#1}}
\newcommand{\AnnotationTok}[1]{\textcolor[rgb]{0.56,0.35,0.01}{\textbf{\textit{#1}}}}
\newcommand{\AttributeTok}[1]{\textcolor[rgb]{0.77,0.63,0.00}{#1}}
\newcommand{\BaseNTok}[1]{\textcolor[rgb]{0.00,0.00,0.81}{#1}}
\newcommand{\BuiltInTok}[1]{#1}
\newcommand{\CharTok}[1]{\textcolor[rgb]{0.31,0.60,0.02}{#1}}
\newcommand{\CommentTok}[1]{\textcolor[rgb]{0.56,0.35,0.01}{\textit{#1}}}
\newcommand{\CommentVarTok}[1]{\textcolor[rgb]{0.56,0.35,0.01}{\textbf{\textit{#1}}}}
\newcommand{\ConstantTok}[1]{\textcolor[rgb]{0.00,0.00,0.00}{#1}}
\newcommand{\ControlFlowTok}[1]{\textcolor[rgb]{0.13,0.29,0.53}{\textbf{#1}}}
\newcommand{\DataTypeTok}[1]{\textcolor[rgb]{0.13,0.29,0.53}{#1}}
\newcommand{\DecValTok}[1]{\textcolor[rgb]{0.00,0.00,0.81}{#1}}
\newcommand{\DocumentationTok}[1]{\textcolor[rgb]{0.56,0.35,0.01}{\textbf{\textit{#1}}}}
\newcommand{\ErrorTok}[1]{\textcolor[rgb]{0.64,0.00,0.00}{\textbf{#1}}}
\newcommand{\ExtensionTok}[1]{#1}
\newcommand{\FloatTok}[1]{\textcolor[rgb]{0.00,0.00,0.81}{#1}}
\newcommand{\FunctionTok}[1]{\textcolor[rgb]{0.00,0.00,0.00}{#1}}
\newcommand{\ImportTok}[1]{#1}
\newcommand{\InformationTok}[1]{\textcolor[rgb]{0.56,0.35,0.01}{\textbf{\textit{#1}}}}
\newcommand{\KeywordTok}[1]{\textcolor[rgb]{0.13,0.29,0.53}{\textbf{#1}}}
\newcommand{\NormalTok}[1]{#1}
\newcommand{\OperatorTok}[1]{\textcolor[rgb]{0.81,0.36,0.00}{\textbf{#1}}}
\newcommand{\OtherTok}[1]{\textcolor[rgb]{0.56,0.35,0.01}{#1}}
\newcommand{\PreprocessorTok}[1]{\textcolor[rgb]{0.56,0.35,0.01}{\textit{#1}}}
\newcommand{\RegionMarkerTok}[1]{#1}
\newcommand{\SpecialCharTok}[1]{\textcolor[rgb]{0.00,0.00,0.00}{#1}}
\newcommand{\SpecialStringTok}[1]{\textcolor[rgb]{0.31,0.60,0.02}{#1}}
\newcommand{\StringTok}[1]{\textcolor[rgb]{0.31,0.60,0.02}{#1}}
\newcommand{\VariableTok}[1]{\textcolor[rgb]{0.00,0.00,0.00}{#1}}
\newcommand{\VerbatimStringTok}[1]{\textcolor[rgb]{0.31,0.60,0.02}{#1}}
\newcommand{\WarningTok}[1]{\textcolor[rgb]{0.56,0.35,0.01}{\textbf{\textit{#1}}}}
\usepackage{graphicx,grffile}
\makeatletter
\def\maxwidth{\ifdim\Gin@nat@width>\linewidth\linewidth\else\Gin@nat@width\fi}
\def\maxheight{\ifdim\Gin@nat@height>\textheight\textheight\else\Gin@nat@height\fi}
\makeatother
% Scale images if necessary, so that they will not overflow the page
% margins by default, and it is still possible to overwrite the defaults
% using explicit options in \includegraphics[width, height, ...]{}
\setkeys{Gin}{width=\maxwidth,height=\maxheight,keepaspectratio}
% Set default figure placement to htbp
\makeatletter
\def\fps@figure{htbp}
\makeatother
\setlength{\emergencystretch}{3em} % prevent overfull lines
\providecommand{\tightlist}{%
  \setlength{\itemsep}{0pt}\setlength{\parskip}{0pt}}
\setcounter{secnumdepth}{-\maxdimen} % remove section numbering
\usepackage{booktabs}
\usepackage{longtable}
\usepackage{array}
\usepackage{multirow}
\usepackage{wrapfig}
\usepackage{float}
\usepackage{colortbl}
\usepackage{pdflscape}
\usepackage{tabu}
\usepackage{threeparttable}
\usepackage{threeparttablex}
\usepackage[normalem]{ulem}
\usepackage{makecell}
\usepackage{xcolor}

\title{LSTAT2130BayesianProject}
\author{Adrien Kinart}
\date{5/5/2021}

\begin{document}
\maketitle

{
\setcounter{tocdepth}{3}
\tableofcontents
}
\begin{Shaded}
\begin{Highlighting}[]
\KeywordTok{library}\NormalTok{(kableExtra)}
\end{Highlighting}
\end{Shaded}

\hypertarget{introduction}{%
\section{Introduction}\label{introduction}}

\begin{Shaded}
\begin{Highlighting}[]
\NormalTok{Table <-}\StringTok{ }\KeywordTok{matrix}\NormalTok{(}\DataTypeTok{data=} \KeywordTok{c}\NormalTok{(}\DecValTok{25}\NormalTok{,}\DecValTok{69}\NormalTok{, }\DecValTok{65}\NormalTok{,}\DecValTok{106}\NormalTok{, }\DecValTok{80}\NormalTok{,}\DecValTok{106}\NormalTok{,}\DecValTok{136}\NormalTok{,}\DecValTok{94}\NormalTok{,}\DecValTok{76}\NormalTok{,}\DecValTok{46}\NormalTok{,}
                       \DecValTok{17}\NormalTok{,}\DecValTok{36}\NormalTok{, }\DecValTok{47}\NormalTok{,}\DecValTok{58}\NormalTok{ , }\DecValTok{47}\NormalTok{,}\DecValTok{53}\NormalTok{ ,}\DecValTok{59}\NormalTok{ ,}\DecValTok{54}\NormalTok{,}\DecValTok{33}\NormalTok{,}\DecValTok{21}\NormalTok{),}
                \DataTypeTok{nrow =} \DecValTok{2}\NormalTok{, }\DataTypeTok{byrow =}\NormalTok{ T)}
\KeywordTok{rownames}\NormalTok{(Table) <-}\StringTok{ }\KeywordTok{c}\NormalTok{(}\StringTok{"Flanders"}\NormalTok{, }\StringTok{"Wallonia"}\NormalTok{) }
\KeywordTok{colnames}\NormalTok{(Table) <-}\StringTok{ }\KeywordTok{c}\NormalTok{(}\StringTok{"<1200"}\NormalTok{, }\StringTok{"[1200-1500)"}\NormalTok{, }\StringTok{"1500-1800"}\NormalTok{, }\StringTok{"1800-2300"}\NormalTok{, }\StringTok{"2300-2700"}\NormalTok{,}
                     \StringTok{"2700-3300"}\NormalTok{, }\StringTok{"3300-4000"}\NormalTok{, }\StringTok{"4000-4900"}\NormalTok{, }\StringTok{"4900-6000"}\NormalTok{, }\StringTok{">6000"}\NormalTok{)}

\NormalTok{NbFlemish <-}\StringTok{  }\KeywordTok{sum}\NormalTok{(Table[}\DecValTok{1}\NormalTok{,])}
\NormalTok{NbWaWalloons <-}\KeywordTok{sum}\NormalTok{(Table[}\DecValTok{2}\NormalTok{,])  }
\end{Highlighting}
\end{Shaded}

\begin{Shaded}
\begin{Highlighting}[]
\NormalTok{kappaFct <-}\StringTok{ }\ControlFlowTok{function}\NormalTok{(phi)\{}\DecValTok{1}\OperatorTok{/}\NormalTok{phi\}}
\NormalTok{lambdaFct <-}\StringTok{ }\ControlFlowTok{function}\NormalTok{(phi,mu)\{}\DecValTok{1}\OperatorTok{/}\NormalTok{(phi}\OperatorTok{*}\NormalTok{mu)\}}
\end{Highlighting}
\end{Shaded}

\begin{verbatim}
## [1] 0.05892352
\end{verbatim}

\hypertarget{question-1}{%
\section{Question 1}\label{question-1}}

Let \(\theta_k:= (\mu_k, \phi_k)\) be the set of parameters for a HNI
with respect to region \(k\).

\hypertarget{a-theoretical-probability}{%
\subsection{(a) Theoretical
probability}\label{a-theoretical-probability}}

Let \(X\) be the monthly net income of 1123 Belgian households net
income (HNI) older than 30 years. Regardless the 2 regions
(\(k=\{1,2\}\) wrt Flanders and Wallonnia, respectively), is assumed it
follows a Gamma distribution. It can be reparametrised in terms of its
mean \(\mu\) and dispersion parameter \(\phi\) with the following trick:

\begin{equation}
\begin{split}
\kappa & = \frac{1}{\phi} \\
\lambda & = \frac{1}{\phi\; \mu}
\end{split}
\end{equation}

For both regions \(k=\{1,2\}\): This gives

\begin{equation}
\begin{split}
f(x_k) = \frac{1}{\Gamma(\phi_k^{-1})} \big( \phi_k \mu \big)^{\phi_k} x_k^{\frac{1}{\phi_k}-1} \exp{(\frac{-x_k}{\phi_k \mu})}
\end{split}
\end{equation}

Then, the probability to fall into a certain HNI interval is:

\begin{equation}
\begin{split}
P(x_{j_1} < x < x_{j_2}) =\int_{x_{j_1}}^{x_{j_2}} \frac{1}{\Gamma(\phi_k^{-1})} \big( \phi_k \mu \big)^{\phi_k} x_k^{\frac{1}{\phi_k}-1} \exp{(\frac{-x_k}{\phi_k \mu})}
\end{split}
\end{equation}

\hypertarget{b-theoretical-expression-for-the-likelihood}{%
\subsection{(b) Theoretical expression for the
likelihood}\label{b-theoretical-expression-for-the-likelihood}}

We have, writing \(P:=(p_{k,1},..,p_{k,10})\) and
\(X_k:= (x_{k,1},... x_{k,10})\) \[
\begin{split}
X_k |P  \sim \text{Mul}(n_k,P)  & = \frac{x_k!}{x_{k,1}! \, ... \, x_{k,10}! } p_1^{n_{x,1}} \times ... \times p_1^{x_{k,10}} \; \text{when} \sum_{j=1}^{10} x_j = x_k \\
&=0 \; \text{otherwise}
\end{split}
\]

\[
\begin{equation}
\begin{split}
L(\theta_k, D_k) = P (D_k | \theta_k) = \pi_{}
\end{split}
\end{equation}
\]

\end{document}
